\documentclass[a4paper,12pt]{article}

\usepackage[french]{babel}
\usepackage[utf8]{inputenc}


\title{Cahier des charges du ProjetNigma}
\author{CrypTeam}
\pagestyle{myheadings}
\date{}				

\begin{document}
	\markright{Cahier des charges de la CrypTeam}
	\maketitle{}
  	\tableofcontents
	\newpage				
	\section{Introduction}
	%introduction rapide
	\newpage
	\section{\'{E}tat de l'art}
		La cryptographie et la stéganographie sont deux techniques extrêmement ancienne permettant de transmettre des informations uniquement aux personnes voulues.

 La cryptographie protège le message en le chiffrant, c'est-à-dire en le rendant incompréhensible sans connaitre l'algorithme de cryptage. On peut citer comme procédé de cryptage historique le chiffre de César qui décale l'alphabet de n rang suivant le chiffre choisi (ainsi si le chiffre choisi est 3, l'alphabet sera : DEFGHI...ZABC).

	La stéganographie consiste à cacher le message à transmettre plutot que de le chiffrer. Comme exemple historique on peut citer un procédé utilisé par César, il écrivait sur le crane d'un esclave un message puis attendait que les cheveux de cet esclave repoussent puis il envoyait l'esclave à la personne à qui le message était destiné. Il suffisait donc de raser l'esclave pour récupérer le message.
		\subsection{Cryptographie}
			\subsubsection{Cryptage symétrique}

			\subsubsection{Cryptage asymétrique}

		\subsection{Stéganographie}
	\newpage
	\section{Répartition des charges}
%descriptif précis	
	\section{Planning de Réalisation} %aligné sur les dates de soutenance	
	\section{Conclusion}
\end{document}